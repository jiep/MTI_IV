\documentclass[12pt,a4paper,twoside,openright,titlepage,final]{article}
\usepackage{fontspec}
\usepackage{amsmath}
\usepackage{amsfonts}
\usepackage{amssymb}
\usepackage{makeidx}
\usepackage{graphicx}
\usepackage[hidelinks,unicode=true]{hyperref}
\usepackage[spanish,es-nodecimaldot,es-lcroman,es-tabla,es-noshorthands]{babel}
\usepackage[left=3cm,right=2cm, bottom=4cm]{geometry}
\usepackage{natbib}
\usepackage{microtype}
\usepackage{ifdraft}
\usepackage{verbatim}
\usepackage[obeyDraft]{todonotes}
\ifdraft{
	\usepackage{draftwatermark}
	\SetWatermarkText{BORRADOR}
	\SetWatermarkScale{0.7}
	\SetWatermarkColor{red}
}{}
\usepackage{booktabs}
\usepackage{longtable}
\usepackage{calc}
\usepackage{array}
\usepackage{caption}
\usepackage{subfigure}
\usepackage{footnote}
\usepackage{url}
\usepackage{tikz}

\setsansfont[Ligatures=TeX]{texgyreadventor}
\setmainfont[Ligatures=TeX]{texgyrepagella}

\input{portada}

\author{José Ignacio Escribano}

\title{Caso práctico IV}

\setlength{\parindent}{0pt}

\begin{document}

\pagenumbering{alph}
\setcounter{page}{1}

\portada{Caso Práctico IV}{Modelización y tratamiento de la incertidumbre}{Modelos lineales}{José Ignacio Escribano}{Móstoles}

\listoffigures
\thispagestyle{empty}
\newpage

\tableofcontents
\thispagestyle{empty}
\newpage


\pagenumbering{arabic}
\setcounter{page}{1}

\section{Introducción}

Este caso práctico buscaremos un modelo lineal para predecir el valor medio de las casas ocupadas por los propietarios, a partir de trece variables como el crimen per cápita, la concentración de óxido nítrico, el número de medio de habitaciones por vivienda, entre otras. 

\section{Resolución del caso práctico}

En primer lugar, observamos el fichero de datos para ver qué variables son cuanitativas y categóricas. Observamos que todas las variables son cuantitativas, excepto una que es categórica: CHAS (1, si las vías cruzan el río y, 0 en caso contrario). Para cada de las variables, calculamos un resumen (mínimo, máximo, primer y tercer cuartil) usando R. El resumen de cada de las variables es el siguiente:

\begin{verbatim}
      CRIM               ZN             INDUS            CHAS        
  Min.   : 0.0060   Min.   :  0.00   Min.   : 0.46   Min.   :0.00000  
  1st Qu.: 0.0820   1st Qu.:  0.00   1st Qu.: 5.19   1st Qu.:0.00000  
  Median : 0.2565   Median :  0.00   Median : 9.69   Median :0.00000  
  Mean   : 3.6135   Mean   : 11.36   Mean   :11.14   Mean   :0.06917  
  3rd Qu.: 3.6770   3rd Qu.: 12.50   3rd Qu.:18.10   3rd Qu.:0.00000  
  Max.   :88.9760   Max.   :100.00   Max.   :27.74   Max.   :1.00000  
       NOX               RM             AGE              DIS        
  Min.   :0.3850   Min.   :3.561   Min.   :  2.90   Min.   : 1.130  
  1st Qu.:0.4490   1st Qu.:5.886   1st Qu.: 45.02   1st Qu.: 2.100  
  Median :0.5380   Median :6.208   Median : 77.50   Median : 3.208  
  Mean   :0.5547   Mean   :6.285   Mean   : 68.57   Mean   : 3.795  
  3rd Qu.:0.6240   3rd Qu.:6.623   3rd Qu.: 94.08   3rd Qu.: 5.189  
  Max.   :0.8710   Max.   :8.780   Max.   :100.00   Max.   :12.126  
       RAD              TAX           PTRATIO            B         
  Min.   : 1.000   Min.   :187.0   Min.   :12.60   Min.   :  0.32  
  1st Qu.: 4.000   1st Qu.:279.0   1st Qu.:17.40   1st Qu.:375.38  
  Median : 5.000   Median :330.0   Median :19.05   Median :391.44  
  Mean   : 9.549   Mean   :408.2   Mean   :18.46   Mean   :356.67  
  3rd Qu.:24.000   3rd Qu.:666.0   3rd Qu.:20.20   3rd Qu.:396.22  
  Max.   :24.000   Max.   :711.0   Max.   :22.00   Max.   :396.90  
      LSTAT            MEDV      
  Min.   : 1.73   Min.   : 5.00  
  1st Qu.: 6.95   1st Qu.:17.02  
  Median :11.36   Median :21.20  
  Mean   :12.65   Mean   :22.53  
  3rd Qu.:16.95   3rd Qu.:25.00  
  Max.   :37.97   Max.   :50.00  
\end{verbatim}

\section{Conclusiones}

\newpage

\section{Código R}

%\verbatiminput{../caso_iv.R}


\end{document} 